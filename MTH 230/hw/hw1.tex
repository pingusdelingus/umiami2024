%  This LaTeX template is based on T.J. Hitchman's work which is itself based on Dana Ernst's template.  
% 
% --------------------------------------------------------------
% Skip this stuff, and head down to where it says "Start here"
% --------------------------------------------------------------
 
\documentclass[12pt]{article}
 
\usepackage[margin=1in]{geometry} 
\usepackage{amsmath,amsthm,amssymb}
\usepackage{graphicx}
\newenvironment{statement}[2][]{\begin{trivlist}
\item[\hskip \labelsep {\bfseries #1}\hskip \labelsep {\bfseries #2.}]}{\end{trivlist}}

\begin{document}
 
% --------------------------------------------------------------
%
%                         Start here
%
% --------------------------------------------------------------
 
\title{MTH 230 HW 1} % replace with the problem you are writing up
\author{Esteban Morales} % replace with your name
\maketitle


\begin{statement}{Excercise 1} %You can use theorem, exercise, problem, or question here.  Modify x.yz to be whatever number you are proving
1. Express each statement using P, Q and logical connectives.
\\
(a) P whenever Q. $ Q => P $\\
(b) P is necessary for Q $ P => Q$\\
(c) P is sufficient for Q. $P => Q$\\
(d) P only if Q. $ Q => P$\\
(e) P is necessary and sufficient for Q $P => Q$\\


\end{statement}



\begin{statement}{Excercise 2}
Prove Pierce's Law: \\
For any proposition P, Q, the proposition$ ((P -> Q) -> P) -> P$ is a
tautology.
\begin{proof}

\begin{displaymath}
\begin{array}{|c c|c|c|c|}
P & Q & P=>Q& (P=>Q)=>P & ((P=>Q)=>P)=>P\\
\hline
T&T&T&T&T\\
T & F & F & T & T\\
F & T & T & F & T\\
F & F & T & F & T\\


\\



\end{array}
\end{displaymath}

\end{proof}

\end{statement}

 

\begin{statement}{Excercise 3}
Prove that $((P=>Q) \land (Q=>R) \land P) => R $ is a tautology
\begin{proof}
\begin{displaymath}
\begin{array}{|c c c|c|c|c|c|}
  P & Q & R & P=>Q& Q=>R & (P=>Q) \land ((Q=>R) & (P=>Q) \land (Q=>R) \land P) 





\end{array}
\end{displaymath}


\end{proof}

\end{statement}
 








% --------------------------------------------------------------
%     You don't have to mess with anything below this line.
% --------------------------------------------------------------
 
\end{document}
