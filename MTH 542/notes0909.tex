
%  This LaTeX template is based on T.J. Hitchman's work which is itself based on Dana Ernst's template.  
% 
% --------------------------------------------------------------
% Skip this stuff, and head down to where it says "Start here"
% --------------------------------------------------------------
 
\documentclass[12pt]{article}
 
\usepackage[margin=1in]{geometry} 
\usepackage{amsmath,amsthm,amssymb}
\usepackage{graphicx}
\newenvironment{statement}[2][Section]{\begin{trivlist}
\item[\hskip \labelsep {\bfseries #1}\hskip \labelsep {\bfseries #2.}]}{\end{trivlist}}

\begin{document}
 
% --------------------------------------------------------------
%
%                         Start here
%
% --------------------------------------------------------------
 
\title{Notes for mth542} % replace with the problem you are writing up
\author{Esteban Morales} % replace with your name
\maketitle


\begin{statement}{testing of hypothesis for a population mean} %You can use theorem, exercise, problem, or question here.  Modify x.yz to be whatever number you are proving
Type 1 Error: Rejecting a true null hypothesis when we should not reject.
Type 2 Error: Failing to reject the null hypothesis when we should reject.\\

If p-value is less than alpha, we reject the null hypothesis since our p-value tells us the likelihood of the observation happening given that the null hypothesis is true.


The further that $H_a$ is from the $H_0$, the smaller the probability of type 2 error as the two distributions are further apart.\\
Increase alpha to reduce probability of type 2 error



\end{statement}


\begin{statement}{testing hypothesis for a population proportion}
For a proportion of a \\
population the idea is similar but execution differs slightly.\\

If $H_0$ were true, that if if p = 0.30, then the chance that at least 34 percent out of a random sample of 500 would watch channel 7 is only 0.0256.
Therefore the data provides strong evidence againt $H_0$. There is strong evidence that the population 

\end{statement}

\begin{statement}{Relationship between two categorical data}
Two categorical variables, want to see if they are associated in any way. We do not use the word "cause" unless we hace a very thorough study.


\end{statement}



% --------------------------------------------------------------
%     You don't have to mess with anything below this line.
% --------------------------------------------------------------
 
\end{document}
